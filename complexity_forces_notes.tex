\documentclass{article}
\usepackage{amsmath, amssymb, color, xcolor, amsthm}
\usepackage{graphicx, wrapfig, float, caption, dsfont, bbm, xfrac, setspace}
\usepackage{fullpage}
\usepackage[backref=page, hidelinks, colorlinks=true, citecolor=blue!60!black!100]{hyperref}
\usepackage{tikz}
\usetikzlibrary{arrows.meta, shapes}
\usepackage{caption, subcaption}
\usepackage{natbib} % gives us \citet: Author (year) and \citep: (Author; year)
\usepackage{authblk}
\usepackage{lineno}
\usepackage{multicol}
\usepackage[bottom]{footmisc}



\begin{document}

\begin{itemize}
\item Describe entropy/ratchet/balance
\item quantify deletion bias
\item quantify entropic force
\item quantify selective pressures
\item implement evolutionary simulation to check above $\uparrow$. 
\item How does equilibrium network size depend on the parameters $p_{\text{mut}}$, $p_{\text{del}}$, $\alpha = p_{\text{new}}/p_{\text{del}}$, $\sigma_{\text{mut}}$, $N$?

\begin{tabular}{l | c | c | r}
\hline
parameter & minimum & default & maximum \\
\hline
$p_{\text{mut}}$ & $10^{-4}$ & $10^{-3}$ & $10^{-1}$ \\
\hline
$p_{\text{del}}$ & $10^{-6}$ & $10^{-4}$ & $10^{-1}$ \\
\hline
$\alpha$ & $10^{-2}$ & $10^{-1}$ & $1$ \\
\hline
$\sigma_{\text{mut}}$ & $10^{-4}$ & & $10^{-1}$ \\
\hline
$N$ & & $1000$ & \\
\hline
\end{tabular}

\item different systems

\item change code to make $p_{\text{add}}$ and $p_{\text{del}}$ per gene instead of per system. 

\item possible experiment: compare equilibrium sizes of systems with a range of large real and negative real eigenvalues. Do systems with large real and positive eigenvalues tend to be smaller due to numerical instability?
\end{itemize}

\paragraph{Deletion bias}

\begin{figure}[H]
\includegraphics[width=0.5\linewidth]{reducibility_plot}
\caption{Network reducibility. Where $\alpha$ between $(0,1)$ on the y-axis, determines equilibrium system size for deletion bias. For instance, at $\alpha = 0.5$, we'd expect system sizes to be around $n=20$.}

\end{figure}

$\beta :=$ probability a deletion is neutral-ish. 

At equilibrium: rate of deletion $\approx$ the rate of new genes. 

$p_{\text{del}} \beta \approx p_{\text{new}} \qquad i.e. \beta \approx \alpha$.

\paragraph{Load}

System size $n$

mutation cost $s$

Strong mutation: $s \approx 1$

Weak selection: mean offspring fitness relative to parent $= s p_{\text{mut}} n^2 = \delta_{\text{add}}$

Where $\delta_{\text{add}}$ is the cost of adding a dimension to a system.

\paragraph{Analytical description}
Can we find a function that, given the current mean size $n_T$ of a population of systems, describes what is the expected size in the next generation/offspring $n_{T+1}$?

\begin{tabular}{l | c | r}
\hline
Offspring size & \% & Mean fitness relative to parent \\
\hline
$n-1$ & $n p_{\text{del}}$ & $1 - s^{\text{DEL}}_{n}$ \\
\hline
$n$ & $1 - (p_{\text{del}} + p_{\text{add}})n$ & $1 - p_{\text{mut}} n^2 s_n$ \\
\hline
$n+1$ & $n p_{\text{add}}$ & $1 - p_{\text{mut}}(n + 1)^2 s_{n+1}$ \\ 
\hline
\end{tabular}
\begin{itemize}
\item Mean fitness of offspring:

$= (1 - p_{\text{mut}} n^2 s_n) + n p_{\text{del}} (1 - s^{\text{DEL}}_{n} - (1 - p_{\text{mut}} n^2 s_n)) + n p_{\text{add}} (1 - p_{\text{mut}}(n + 1)^2 s_{n+1}) - (1 - p_{\text{mut}} n^2 s_n)$ 

$= 1 - p_{\text{mut}} n^2 s_n - n p_{\text{del}} (s^{\text{DEL}}_{n} - p_{\text{mut}}n^2 s_n) - n p_{\text{add}} p_{\text{mut}} ((n+1)^2 s_{n+1} - n^2 s_n)$

\item Mean network size in offspring

$ = \frac{\sum_{\text{sizes}} (\% \text{size}) \times (\text{fitness at size}) \times \text{size})}{\sum_{\text{sizes}} (\% \text{size}) \times (\text{fitness at size})}$

$= n + \frac{n p_{\text{add}} (1 - p_{\text{mut}}(n+1)^2 s_{n+1}) - n p_{\text{del}} (1 - s^{\text{DEL}}_{n})}{\bar{w}}$

$= \frac{(n-1) a_{n-1} + n a_n + (n+1) a_{n+1}}{a_{n-1} + a_n + a_{n+1}} = n + \frac{a_{n+1} - a_{n-1}}{a_{n+1}+a_n + a_{n-1}}$

\item So: network size will go up if this is $>n$,

that is if

$\not n p_{\text{add}} (1 - p_{\text{mut}}(n+1)^2 s_{n+1}) > \not n p_{\text{del}} (1 - s_n^{\text{\text{del}}})$

that is,

$\alpha = \frac{p_{\text{add}}}{p_{\text{del}}} > \frac{1 - s^{\text{DEL}}_{n}}{1 - p_{\text{mut}}(n-1)^2 s_{n+1}}$

$\approx 1 + p_{\text{mut}} (n+1)^2 s_{n+1} - s^{\text{DEL}}_{n}$

To be more exact:

$\alpha > \frac{1 -s^{\text{DEL}}_{n}}{1 - p_{\text{mut}} ((n+1)^2 s_{n+1} - (n-1)^2 s_{n-1})}$

$\approx \frac{1 - s^{DEL}_{n}}{1 - p_{\text{mut}} 4ns_n}$
%$\approx 1 + p_{\text{mut}} 4ns_n - s^{\text{DEL}}_{n}$.
\end{itemize}


\begin{equation}
\mathbb{E} \left[ n_{t+1} \right] \approx n_t \left( 1 + \frac{p_{\text{add}} \left(1 - p_{\text{mut}}4ns_n\right) - p_{\text{del}} \left(1 - s^{\text{DEL}}_{n}\right)}{\bar{w}} \right)
\end{equation}

For any type of system (e.g. oscillator), we need to find empirically, $s^{\text{DEL}}_n$ and $s_n$. We also need to know how this changes with $\sigma_{\text{mut}}$. 

What is the math for mutational meltdown? 




\end{document}
